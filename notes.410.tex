\documentclass[12pt,letterpaper]{article}

% \usepackage[margin=0.9in, showframe]{geometry}
\usepackage[
    top=5mm,
    bottom=5mm,
    left=5mm,
    right=5mm,
    marginparwidth=0mm,
    marginparsep=0mm,
    headheight=15pt,
    centering,
    % showframe,
    includefoot,
    includehead
    ]{geometry}


\usepackage{setspace}
\usepackage{titlesec}
\usepackage{lipsum}
\usepackage{amssymb}
\usepackage{amsmath}
\usepackage{xcolor}

\usepackage{blindtext}
\usepackage{multicol}
\usepackage{fancyhdr}

% add more margin between multicols columns
\setlength\columnsep{10pt} 

\vspace{-\baselineskip}

% ================== %
% format of sections
% ================== %
\titleformat{\section}
   {\normalfont\small\bfseries\MakeUppercase}{}{1em}{\underline}
\titlespacing{\section}{-1pc}{*4}{*1.5}

\titleformat{\subsection}
   {\normalfont\normalsize\bfseries}{}{1em}{}
\titlespacing{\subsection}{-0.7pc}{*4}{*1.5}

\titleformat{\subsubsection}
   {\small\bfseries\sc}{}{1em}{}
\titlespacing{\subsection}{-0.5pc}{*4}{*1.5}



% ================= %
% Headers & Footers
% ================= %
\pagestyle{fancy}
\fancyhf{}
\lhead{\bf STAT410 Notes}
\rhead{Gaetan Almela}
\rfoot{\thepage}
\renewcommand{\headrulewidth}{0pt}



% ================= %
% Aliases
% ================= %
\def\cov{\text{Cov}}
\def\var{\text{V}}
\def\ev{\text{E}}
\def\eps{\varepsilon{}}
\def\Dom{\text{Dom}}



% ================= %
% Commands
% ================= %
\newcommand{\Z}{{\mathbb Z}}
\newcommand{\Q}{{\mathbb Q}}
\newcommand{\R}{{\mathbb R}}
\newcommand{\D}{{\mathbb D}}
\newcommand{\N}{{\mathbb N}}

\newcommand{\btw}[1]{
    $\langle$ #1 $\rangle$
}

\newcommand{\powerset}{\raisebox{.15\baselineskip}{\Large\ensuremath{\wp}}}

% \newcommand{\B}{\raisebox{.15\baselineskip}{\Large\ensuremath{\wb}}}

% create a command to display examples in a box
\newcommand{\ex}[1]{
    \medskip \noindent
    \fcolorbox{lightgray}{white}{
        \parbox{0.8\linewidth}{
            {\bf Example} \\

            #1
        }
    } 
}

% create a command for TODOs
\newcommand{\TODO}{\color{red}\textbf{TODO}\color{black}}



\begin{document}
    \footnotesize

    \begin{multicols*}{2}
        \btw{Zoom recordings and notes are available through elms}

        \section{Probabilities}

        Goal is to understand how to calculate probabilities.

        \subsection{Experiments}

        An experiment is {\it a task that has 2 attributes}:
        \begin{enumerate}
            \item It is repeatable
            \item All outcomes are known and well defined.
        \end{enumerate}

        \ex {
            Pick out colored balls from a bag, you need to know what the different
            colors can be.

            Once an experiment has been designed, you will have the {\bf sample
            space $S$}: the set of all possible outcomes of the experiment.
        }

        \ex {
            Experiment: Toss a coin once.

            Sample Space: $\{H, T\}$. We assume that this is a two-sided coin.
        }

        \ex {
            Toss that same coin 4 times.

            Sample Space

            \[
                S = \begin{bmatrix}
                    HHHH & HHHT & HHTH & HHTT \\

                    HTHH & HTHT & HTTH & HTTT \\

                    THHH & THHT & THTH & THTT \\

                    TTHH & TTHT & TTTH & TTTT
                \end{bmatrix}
            \]

            There are 16 outcomes in this sample space. $|S| = 16$. Note that
            this space is 1 dimensional, I only formatted it as a matrix to save
            space.
        }

        \ex {
            Roll a 6-sided die once

            \[
                S = \begin{bmatrix} 
                    1 \\
                    2 \\
                    \vdots \\
                    5 \\
                    6
                \end{bmatrix}
            \]

            Roll a 6-sided die twice

            \[
                S = \begin{bmatrix} 
                    (1, 1) & \dots  & (1, 6) \\
                    \vdots & \ddots & \vdots \\
                    (6, 1) & \dots  & (6, 6) 
                \end{bmatrix}
            \]
        }

        \subsection{Events}
        An event is any subset $E$ of the sample space.

        \ex {
            \[
                E = \begin{bmatrix} 
                    (1, 1) \\
                    (2, 2) \\
                    \vdots \\
                    (5, 5) \\
                    (6, 6)
                \end{bmatrix}
            \]
        }

        \subsection{Probabilities of Events}

        There are $\powerset(S) = 2^{|S|}$ different subsets $E$ of $S$. This
        grows very quickly. In our case of rolling a die twice, the set of all
        subsets is $\powerset(S) = 2^{36}$.

        {\bf Note}: For a set $F$, $F^c = U - F$ is the complement of $F$
        where $U$ is the universe.

        \ex {
            For a set $F = \{x \leq 10\}$, $F^c = \{ x > 10 \}$
        }

        \subsection{Set Theory}

        Let $A$ and $B$ be sets. \\

        {\bf Complement}

        $A^c = \{ x \not\in A \}$ is the complement of $A$. \\

        {\bf Subsets}

        $A \subseteq B$ if and only if $x \in A \Rightarrow x \in B$ \\

        {\bf Union}

        $A \cup B = \{x | x \in A \text{ or } x \in B\}$ \\

        {\bf Intersections}

        $A \cap B = \{x | x \in A \text{ and } x \in B\}$ \\

        \btw{Goal: Quantify the probability of an event.}

        It is important to note that one should consider whether the events in
        the experiment are dependent or independent.

        \ex {
            Suppose the experiment is to roll a die and weight it. Then the
            sample space $S = \{ \text{weight of the die $\in \R$} \}$. This is
            true regardless of the number shown on the die, since these two
            values are {\it independent}.

            Note that $\R$ is uncountably infinite
        }

        If the sample space $S$ is finite, and $|S| = N$, any subset of $S$ is
        an event. There are $\powerset(S) = 2^N$ possible subsets.

        \ex {
            {\bf Toss a coin once.}

            Then $S = \{H, T\}$ so $|S| = 2$, then $|\powerset(S)| = 2^2 = 4$.
            \\

            {\bf Toss a coin twice.}

            Then $|S| = 4$, then $|\powerset(S)| = 2^4 = 16$.
            \\

            {\bf Roll a die once.}

            Then $|S| = 6$, then $|\powerset(S)| = 2^6 = 64$.
            \\

            {\bf Roll a die twice.}

            Then $|S| = 36$, then $|\powerset(S)| = 2^{36}$.
            \\

            {\bf Roll an $N$ sided die.}

            Then $|S| = N$, then $|\powerset(S)| = 2^N$.
            \\
        }

        {\bf Note} Event space is growing exponentially in the size of the
        sample space

        \section{$\sigma$ Algebras}

        \btw {{\bf Goal} Cut down the event space based on interesting/important
        events.}

        We will define probabilities only for events we are interested in.

        A collection of $B$ sets is called a sigma algebra if
        \begin{itemize}
            \item $\emptyset \in B$
            \item If $A \in B$, then $A^c \in B$ ($B$ is closed under
                complements)
            \item If $A_1, A_2, A_3, ...$ is a countable collection of sets
                in $B$, then $\bigcup_{i = 1} A_i \in B$.
        \end{itemize}

        \ex {
            Toss the coin three times.

            $\powerset(S)$ is the largest possible sigma algebra with size
            $2^|S| = 2^8 = 256$.
        }

        \ex {
            Toss the coin three times, is the second toss a $H$?

            $|S| = 2^3 = 8$ but now we only care about 2 types of outcomes,
            either the middle toss is $H$ or $T$. \\

            Let $\{0\}$ be the set such that you toss a coin 3 times and second
            toss is $H \rightarrow \{HHH, HHT, THT, THH\}$

            Let $\{1\}$ be the set such that you toss a coin 3 times and second
            toss is $T \rightarrow \{TTT, TTH, HTT, HTH\}$ \\

            Let $B$ be the sigma algebra generated by $\{0\}$ and $\{1\}$, then

            \TODO {Format better}
            \begin{align*}
                B&= \{ \emptyset, \{0, 1\}, \{0\}, \{1\}\} \\
                 &= \{ \emptyset, S, \\
                 &      \{HHH, HHT, THH, THT\}, \\
                 &      \{TTT, TTH, HTT, HTH\} \\
                 &  \}
            \end{align*}

            For step 2, we just substituted $\{0\}$ and $\{1\}$.

            Note that the union of any two sets in $B$ is also in $B$. 
        }

        Note that the reason we don't just use $\powerset(S)$ all the time for
        probability is because the powerset gets huge so fast, even for
        computers. This is the problem that sigma algebras solve.

        This is useful to define a probability function. \\

        \subsection{Probability Function}
        Given a pair $(S, B)$ where $S$ is the sample space and $B$ is the sigma
        algebra. A {\bf probability function} $P: B \rightarrow \R$ for this
        pair is a function on $B$ satisfying 3 axioms

        \begin{itemize}
            \item $P(\emptyset) = 0$.
            \item $P(S) = 1$.
            \item $P(A) \geq 0$ for all $A \in B$.
            \item If $A_1, A_2, A_3, ...$ is a collection of pairwise disjoint
                sets in $B$, $P(\bigunion_{i} A_i) = \sum_{i = 1}^{\infty}
                P(A_i)$. Note that $S$ could be infinite, which means that since
                $B \subseteq \powerset(S)$ $A_i$s could be infinite.
        \end{itemize}

    \end{multicols*}
\end{document}

